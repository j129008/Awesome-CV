%-------------------------------------------------------------------------------
%	SECTION TITLE
%-------------------------------------------------------------------------------
\cvsection{Thesis}


%-------------------------------------------------------------------------------
%	CONTENT
%-------------------------------------------------------------------------------
\begin{cventries}
%---------------------------------------------------------
  \cventry
    {Master Paper} % Organization
    {Sentence Segmentation for Tomb Biographies of Tang Dynasty and Chinese Buddhist Temple Gazetteers} % Job title
    {National Chengchi University} % Location
    {2018} % Date(s)
    {
      \begin{cvitems} % Description(s) of tasks/responsibilities
        \item {Tools:Keras、sklearn-crfsuite}
        \item {Git:https://github.com/j129008/paper-work}
      \end{cvitems}
    }

%---------------------------------------------------------

  \cventry
    {PACLIC(Pacific Asia Conference on Language, Information and Computation)} % Organization
    {An Analysis of the Style of the Tang Poetry and the Social Network of the Poets}
    {Shanghai} % Location
    {2015} % Date(s)
    {
      \begin{cvitems} % Description(s) of tasks/responsibilities
        \item {Presenter} % Event
        \item {paper:http://bcmi.sjtu.edu.cn/home/paclic29/proceedings/PACLIC29-2016.86.pdf}
      \end{cvitems}
    }

%---------------------------------------------------------

  \cventry
    {DH(DIGITAL HUMANITIES)} % Organization
    {Classical Chinese Sentence Segmentation for Tomb Biographies of Tang Dynasty} % Event
    {Mexico} % Location
    {2018} % Date(s)
    {
      \begin{cvitems} % Description(s) of tasks/responsibilities
      \item {2nd author} % Event
      \item {
            \href {https://dh2018.adho.org/classical-chinese-sentence-segmentation-for-tomb-biographies-of-tang-dynasty/}{paper link}
         }
      \end{cvitems}
    }

%---------------------------------------------------------
\end{cventries}
